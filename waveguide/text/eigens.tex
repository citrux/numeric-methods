\documentclass[a4paper,12pt]{article}
\usepackage[russian]{babel}
\usepackage[utf8]{inputenc}
\usepackage[sumlimits, intlimits]{amsmath}
\usepackage{amssymb}
\usepackage{tikz, pgfplots}
\usepackage[margin=1.5cm]{geometry}
\begin{document}
\section{Дифференциальные и матричные операторы. Собственные значения}
\subsection{Дифференциальный оператор}
	Рассмотрим дифференциальный оператор
	\[
		\frac{\partial^2}{\partial x^2}
	\]
	и определим его собственные значения \( \lambda_k \) и собственные функции
	\( \phi_k(x) \) при дополнительном граничном условии
	\[
		\phi_k(0) = \phi_k(1) = 0.
	\]
	Задача на собственные значения имеет вид:
	\[
		\phi_k'' = \lambda_k\phi_k,\quad
		\phi_k(x) = c_1e^{-ig_k x} + c_2e^{ig_k x},
	\]
	где \( g_k = \sqrt{-\lambda_k} \). Воспользуемся дополнительным граничным
	условием:
	\[
		c_1 + c_2 = c_1e^{-ig_k} + c_2e^{ig_k} = 0,
	\]
	\[
		c_2 = -c_1 = c \Rightarrow \sin g_k = 0 \Rightarrow g_k = k\pi,
		k\in\mathbb{N}.
	\]
	В итоге
	\[
		\lambda_k = -k^2\pi^2,\quad \phi_k = \sin k\pi x,\quad k\in\mathbb{N}.
	\]

\subsection{Матричный оператор}
	Перенесёмся теперь из пространства непрерывных функций в пространство сеточных. Разобьём \( [0,1] \) на \(n\) частей, при этом функция будет определяться своими значениями в \( n+1 \) точках этого отрезка. Таким образом, функция становится вектором \( n+1 \)-мерного пространства.

	Теперь определимся с тем, во что переходит оператор \( \frac{\partial^2}{\partial x^2} \). Нас интересуют только значения этого оператора в узла разбиения, поэтому разложим функцию в соседних узлах в ряд Тейлора и посмотрим, что получится:

	\begin{align*}
		& f(x_{k+1}) = f(x_k) + f'(x_k)h + \frac{f''(x_k)}{2}h^2 + \frac{f^{III}(x_k)}{6}h^3 + o(h^4),\\
		& f(x_{k-1}) = f(x_k) - f'(x_k)h + \frac{f''(x_k)}{2}h^2 - \frac{f^{III}(x_k)}{6}h^3 + o(h^4),\\
	\end{align*}
	откуда
	\[
		f''(x_k) = \frac{f(x_{k-1}) - 2f(x_k)  + f(x_{k+1})}{h^2} + o(h^2).
	\]

	Но вспомним теперь, что мы можем рассматривать функции как векторы, поэтому это же можно переписать в виде
	\[
		f_k'' = D^2_{km}f_m,\quad D^2_{km} = \begin{cases}
			1/h^2, \quad \text{если } |m-k| = 1,\\
			-2/h^2, \quad \text{если } m=k,\\
			0, \quad \text{иначе}.
		\end{cases}
	\]
	Тут возникает естественная проблема с границами, так как не существует \( f_{-1} \) и \( f_{n+1} \). В них вторую производную с точностью \( o(h^2) \) можно выразить иначе:
	\begin{align*}
		& f_0'' = \frac{2f_0 - 5f_1 + 4f_2 - f_3}{2h^2},\\
		& f_n'' = \frac{2f_n - 5f_{n-1} + 4f_{n-2} - f_{n-3}}{2h^2},\\
	\end{align*}

	Оператор в этом случае принимает вид:
	\[
		D^2 = \frac{1}{2h^2}
		\begin{pmatrix}
			2 & -5 &  4 & -1 & 0 & \cdots & 0 & 0 & 0 & 0 \\
			2 & -4 &  2 &  0 & 0 & \cdots & 0 & 0 & 0 & 0 \\
			0 &  2 & -4 &  2 & 0 & \cdots & 0 & 0 & 0 & 0 \\
			0 &  0 &  2 & -4 & 2 & \cdots & 0 & 0 & 0 & 0 \\
			\vdots&\vdots&\vdots&\vdots&\vdots&\ddots&\vdots&\vdots&\vdots&\vdots\\
			0 &  0 &  0 &  0 & 0 & \cdots &  2 & -4 &  2 & 0 \\
			0 &  0 &  0 &  0 & 0 & \cdots &  0 &  2 & -4 & 2 \\
			0 &  0 &  0 &  0 & 0 & \cdots & -1 &  4 & -5 & 2 \\
		\end{pmatrix}
	\]
	
\subsection{Собственные значения и собственные векторы}
	Задача на собственнные значения оператора переходит в задачу на собственные значения матрицы:
	\[
		D^2 f_k = \mu_k f_k.
	\]

	Сразу перейдём к рассмотрению функций, удовлетворяющих граничному условию
	\[
		f_0 = f_n = 0.
	\]
	В этом случае вместо \( n+1 \)-мерного вектора можно перейти к \( n-1 \)-мерному. Оператор при этом примет вид:
	\[
		D^2 = \frac{1}{h^2}
		\begin{pmatrix}
			-2 & 1 &  0 &  0 & \cdots & 0 & 0 & 0\\
			1 & -2 &  1 &  0 & \cdots & 0 & 0 & 0\\
			0 &  1 & -2 &  1 & \cdots & 0 & 0 & 0\\
			\vdots&\vdots&\vdots&\vdots&\ddots&\vdots&\vdots&\vdots\\
			0 &  0 &  0 &  0 & \cdots &  1 & -2 &  1 \\
			0 &  0 &  0 &  0 & \cdots &  0 &  1 & -2 \\
		\end{pmatrix}
	\]
	Так как мы не рассматриваем \(f_0\) и \(f_n\), то из оператора уходят первая и последняя строки, а учёт их нулевых значений позволяет убрать первый и последний столбцы.

	Зная явный вид оператора можно найти его собственные значения и векторы и сравнить с соответствующими значениями для дифференциального оператора:
	\begin{center}
		\tikz[anchor=center,baseline] \node{
		\begin{tikzpicture}
		\draw (3,6) node{\small $\mu_1 = -9.852449,\quad\lambda_1=-9.869604$};
		\begin{axis}[
			axis x line=center,
			ymajorgrids,
			axis y line=left,
			enlargelimits=0.05]
			\addplot coordinates {
				(0.00,0)
				(0.05,0.049)
				(0.10,0.098)
				(0.15,0.144)
				(0.20,0.186)
				(0.25,0.224)
				(0.30,0.256)
				(0.35,0.282)
				(0.40,0.301)
				(0.45,0.312)
				(0.50,0.316)
				(0.55,0.312)
				(0.60,0.301)
				(0.65,0.282)
				(0.70,0.256)
				(0.75,0.224)
				(0.80,0.186)
				(0.85,0.144)
				(0.90,0.098)
				(0.95,0.049)
				(1.00,0)};
			\addplot[domain=0:1,samples=100,line width=1.5pt]{0.316 * sin(deg(3.14 * x))};
			\end{axis}
		\end{tikzpicture}
		};\hfill
		\tikz[anchor=center,baseline] \node{
		\begin{tikzpicture}
		\draw (3,6.4) node{\small $\mu_2 = -39.154787,\quad\lambda_2=-39.478418$};
		\begin{axis}[
			axis x line=center,
			ymajorgrids,
			axis y line=left,
			enlargelimits=0.05]
			\addplot coordinates {
				(0.00,0)
				(0.05,0.098)
				(0.10,0.186)
				(0.15,0.256)
				(0.20,0.301)
				(0.25,0.316)
				(0.30,0.301)
				(0.35,0.256)
				(0.40,0.186)
				(0.45,0.098)
				(0.50,0.000)
				(0.55,-0.098)
				(0.60,-0.186)
				(0.65,-0.256)
				(0.70,-0.301)
				(0.75,-0.316)
				(0.80,-0.301)
				(0.85,-0.256)
				(0.90,-0.186)
				(0.95,-0.098)
				(1.00,0)};
			\addplot[domain=0:1,samples=100,line width=1.5pt]{0.316 * sin(deg(6.28 * x))};
			\end{axis}
		\end{tikzpicture}};
	\end{center}
	Способ их определения описан в следующих разделах.

	Найдём также оператор для граничного условия вида
	\[
		f_0' = f_n' = 0.
	\]
	Для этого добавим с каждого из краёв по виртуальному узлу: \( f_{-1} \) и \( f_{n+1} \). Тогда из равенства нулю производной на краю через среднюю производную находим
	\[
		f_{-1} = f_{1},\quad f_{n-1} = f_{n+1}.
	\]

	Используя виртуальные узлы, вторую производную на границах можно выразить так
	\[
		f_0'' = \frac{f_{-1} - 2 f_0 + f_1}{h^2} = \frac{-2f_0 + 2 f_1}{h^2}, \quad
		f_n'' = \frac{f_{-1} - 2 f_0 + f_1}{h^2} = \frac{2 f_{n-1}-2f_n}{h^2}.
	\]
	Матрица оператора принимает вид
	\[
		D^2 = \frac{1}{h^2}
		\begin{pmatrix}
			-2 & 2 &  0 &  0 & 0 & \cdots & 0 & 0 & 0 & 0 \\
			1 & -2 &  1 &  0 & 0 & \cdots & 0 & 0 & 0 & 0 \\
			0 &  1 & -2 &  1 & 0 & \cdots & 0 & 0 & 0 & 0 \\
			0 &  0 &  1 & -2 & 1 & \cdots & 0 & 0 & 0 & 0 \\
			\vdots&\vdots&\vdots&\vdots&\vdots&\ddots&\vdots&\vdots&\vdots&\vdots\\
			0 &  0 &  0 &  0 & 0 & \cdots &  1 & -2 &  1 & 0 \\
			0 &  0 &  0 &  0 & 0 & \cdots &  0 &  1 & -2 & 1 \\
			0 &  0 &  0 &  0 & 0 & \cdots &  0 &  0 &  2 & -2 \\
		\end{pmatrix}
	\]
\end{document}